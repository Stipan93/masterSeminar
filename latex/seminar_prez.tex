%%%%%%%%%%%%%%%%%%%%%%%%%%%%%%%%%%%%%%%%%
% Beamer Presentation
% LaTeX Template
% Version 1.0 (10/11/12)
%
% This template has been downloaded from:
% http://www.LaTeXTemplates.com
%
% License:
% CC BY-NC-SA 3.0 (http://creativecommons.org/licenses/by-nc-sa/3.0/)
%
%%%%%%%%%%%%%%%%%%%%%%%%%%%%%%%%%%%%%%%%%

%----------------------------------------------------------------------------------------
%	PACKAGES AND THEMES
%----------------------------------------------------------------------------------------

\documentclass{beamer}

\mode<presentation> {

% The Beamer class comes with a number of default slide themes
% which change the colors and layouts of slides. Below this is a list
% of all the themes, uncomment each in turn to see what they look like.

%\usetheme{default}
%\usetheme{AnnArbor}
%\usetheme{Antibes}
%\usetheme{Bergen}
%\usetheme{Berkeley}
%\usetheme{Berlin}
\usetheme{Boadilla}
%\usetheme{CambridgeUS}

}
\usepackage{algorithmic}
\usepackage{algorithm}
\usepackage{booktabs}
\usepackage{listings}
\usepackage{xcolor}
\newcommand\myworries[1]{\textcolor{red}{#1}}
\usepackage{hyperref} 
\usepackage{natbib}
\usepackage{subfig}
\usepackage{color}
\usepackage{placeins}
\usepackage[utf8]{inputenc}
\usepackage{graphicx} % Allows including images
\usepackage{booktabs} % Allows the use of \toprule, \midrule and \bottomrule in tables


%\setbeameroption{show notes}
%----------------------------------------------------------------------------------------
%	TITLE PAGE
%----------------------------------------------------------------------------------------


\title[Seminar]{Medujezično prepoznavanje imenovanih entiteta pomoću wikifikacije} % The short title appears at the bottom of every slide, the full title is only on the title page

\author{Stipan Mikulić} % Your name
\institute[FER]{
\textit{Mentor: doc. dr. sc. Jan Šnajder}\\
\medskip
Fakultet elektortehnike i računarstva \\
Sveučilište u Zagrebu \\
}
\date{1. lipnja 2017.}

\begin{document}



\begin{frame}
\titlepage % Print the title page as the first slide
\end{frame}

\note[]{
Dobar dan svima. Ja sam Stipan Mikulić. Tema moga završnog rada je primjena strojnog učenja za tematsku analizu sentimenta.
}

\begin{frame}
\frametitle{Sadržaj} % Table of contents slide, comment this block out to remove it
% \tableofcontents % Throughout your presentation, if you choose to use \section{} and \subsection{} commands, these will automatically be printed on this slide as an overview of your presentation
\end{frame}

\note[]{
Prvo ćemo napraviti pregled sadržaja. Reći ću nešto o prikupljenim podatcima nad kojima je radena analiza.
Nakon toga ću predstaviti probleme detekcije ili otkrivanja teme i analize sentimenta korištene u ovom radu te reći nešto više o razvijenim modelima strojnog učenja.
}



%----------------------------------------------------------------------------------------
%	PRESENTATION SLIDES
%----------------------------------------------------------------------------------------









\begin{frame}
\begin{center}
\Huge Hvala na pažnji! \\
 Pitanja?
\end{center}
\end{frame}
\end{document} 